\clearpage % clear the prior chapter's page

\chapter{Other Latex References}\label{CH3_AIM1}
%\vspace{-7mm}
%\bigskip


\section{Basics of Labels and Referencing}

Since we can use any string as a label, it's a common practice to add a few letters to the label (as prefix) to indicate what is being labeled. This becomes important when a lot of different types of objects are referenced in a document, as it might be useful to remember the kind of object a label refers to it. Besides, it also makes it possible to reference different kind of objects using a common string.

For example, if a document contains

a section on population of nations,
a table containing the population numbers, and
a figure containing a bar chart of the population sizes,
it might be convenient to refer to all of them using variants of population. This can be accomplished by using the labels

sec:population for the section,
tab:population for the table, and
fig:population for the figure.
The following is an example for figures -

\begin{verbatim}
\begin{figure}[h!]
  \includegraphics[scale=1.7]{birds.jpg}
  \caption{The birds}
  \label{fig:birds}
\end{figure}
Again, note that \label is given after \caption.
\end{verbatim}

 This is an example of a reference to Figure\ref{fig:logo}


\begin{figure}[h]
\centering
\includegraphics[width=6cm]{mcs_stacked.png}
 \caption{MSCS logo}
 \label{fig:logo}
\end{figure}

%%%%%%%%%%%%%%%%%%%%%%%%%%%%%%%%%%%%%%%%%%%%%%%%%%%%
\subsection{Equation Reference}
All equations should be numbered, either with a single equation number 
or with a section number and an equation number separated by a 
dash. Equations should appear centered, inline with the text but 
with extra space above and below. Equation numbers 
should appear on the first line of each equation, right-justified 
and in parentheses. Variables should be written in italic both in 
equations and in the text.

\begin{equation}
\label{eq:1}
\sum_{i=0}^{\infty} a_i x^i
\end{equation}

The equation \ref{eq:1} shows a sum that is divergent. This formula 
will used in Chapter \ref{CH2_RelatedWork}.

For further references see \href{http://www.overleaf.com}{Something 
Linky} or go to the next url: \url{http://www.overleaf.com}.

%%%%%%%%%%%%%%%%%%%%%%%%%%%%%%%%%%%%%%%%%%%%%%%%%%%%
\section{IEEE Citation Style}

All references in the bibliography should be in IEEE format.  

Please practice adding a reference via the following steps.  

The following citations are examples using the IEEE style. 
 Items with same author is shown in "---"\\
 \\

\noindent An article \cite{anarticle}\\
A book \cite{abook}\\
A series \cite{bookseries}\\
Someone's thesis \cite{thesis}\\
Some technical report \cite{report}\\
A collection \cite{collection}\\
Visited website \cite{website}\\
Accepted for publication \cite{acceptedpub}\\
Submitted for publication \cite{unpub}\\
Not published \cite{notpub}\\
Conversation \cite{conv}\\

%%%%%%%%%%%%%%%%%%%%%%%%%%%%%%%%%%%%%%%%%%%%%%%%%%%%
\section{Sample Tables}

Table\ref{table:test} is defined in another file \href{run:./tableTest.tex}{tableTest.txt}. 

\begin{table}[h!]
\scriptsize
\renewcommand{\tabcolsep}{0.09cm}
\centering
\input{tableTest.tex}
\caption{A sample table.}
\label{table:test}
\end{table}

Table\ref{table:test2} is define in the same file. 

% Table generated by Excel2LaTeX from sheet 'Sheet1'

\begin{table}[h!]
\centering
\begin{tabular}{rrcc}
\addlinespace
\toprule
\multicolumn{ 2}{c}{{\bf Condition}} & {\bf Metric I} & {\bf Metric II} \\
\otoprule
{\bf } & {\bf Mean} & 1505.644 & 1428.076 \\
{\bf Method A} & {\bf Std} & 726.160 & 541.098 \\
{\bf } & {\bf Reduction} & 0.000 & 0.000 \\
\midrule
{\bf } & {\bf Mean} & 1490.841 & 1426.620 \\
{\bf Method B} & {\bf Std} & 735.995 & 543.489 \\
{\bf } & {\bf Reduction} & 14.803 & 1.456 \\
\midrule
{\bf } & {\bf Mean} & 591.843 & 458.001 \\
{\bf Method C} & {\bf Std} & 458.332 & 153.099 \\
{\bf } & {\bf Reduction} & 913.801 & 970.075 \\
\midrule
{\bf } & {\bf Mean} & 566.089 & 638.568 \\
{\bf Method D} & {\bf Std} & 701.194 & 304.485 \\
{\bf } & {\bf Reduction} & 939.555 & 789.508 \\
\midrule
{\bf } & {\bf Mean} & 242.422 & 186.369 \\
{\bf Method E} & {\bf Std} & 390.052 & 129.654 \\
      & {\bf Reduction} & 1263.222 & 1241.707 \\
\midrule
\end{tabular}
\caption{Table to test captions and labels.}
\label{table:test2}
\end{table}

For further references see \href{https://www.overleaf.com/learn/latex/Tables}{Something 
Linky} or go to the next url: \url{https://www.overleaf.com/learn/latex/Tables}.

%%%%%%%%%%%%%%%%%%%%%%%%%%%%%%%%%%%%%%%%%%%%%%%%%%%%
\subsection{Second Level Heading}